\documentclass[letterpaper]{article}
\usepackage{amsmath}
\usepackage{amssymb}
\usepackage{xcolor}
\usepackage{fancyvrb}
\usepackage{graphicx}
\usepackage{helvet}
\usepackage{newtxsf}

\graphicspath{ {../images/} }
\definecolor{mBlue}{rgb}{0,0.5,1}
\definecolor{mGray}{rgb}{0.5,0.5,0.5}
\definecolor{mPurple}{rgb}{0.58,0,0.82}
\definecolor{backgroundColour}{rgb}{0.95,0.95,0.95}
\definecolor{lBlue}{rgb}{0.7,0.8,1.0}
%TODO Title
\def\T{Chapter 2}
%TODO Day
\def\D{8}
%TODO Month
\def\M{1}
%TODO Year
\def\Y{2024}
\addtolength\parskip{\smallskipamount}
\addtolength{\oddsidemargin}{-.875in}
\addtolength{\evensidemargin}{-.875in}
\addtolength{\textwidth}{1.5in}
\addtolength{\topmargin}{-4in}
\setlength{\topmargin}{0pt}
\title{\vspace{-3.0cm}\T}
\date{\vspace{-1cm}\D \ / \M  \ / \Y}

\renewcommand{\familydefault}{\sfdefault}

\begin{document}

\maketitle

\section*{The Peano Axioms (God help us)}
\textbf{The natural numbers are the elements of the set $N := \{0,1,2,3,4,\dots\}$}

\subsection*{The Axioms}
\begin{enumerate}
	\item 0 is a natural number
	\item if $n$ is a natural number, then $n++$ is also a natural number
	\item $\forall n \in N, n++ \neq 0$
	\item $\forall n,m \in N, n++ = m++ \to n=m$
	\item \emph{Induction:} $P(n)$ is a property of a natural number $n$, $P(0) \land P(n++) \to P(n) \text{ is true for all } n \in N$
\end{enumerate}

\subsection*{Induction}
At least while we only look at the natural numbers, proving $P(0)$, assuming $P(n) \forall n\in N$ and proving $P(n++)$ proves $P(n)$.

From what I get, we can do this because once $P(0) \land P(n++)$, $P(0 ++ \text{indefinite times})$ becomes true.

\section*{Recursive definitions}
Recursive definitions generate a sequence based on a function. An initial value, $a_0$ is defined, and the rest of the sequence is defined as $a_{n++} := f(a_{n})$. Each element of the sequence will now have a unique value, which in turn produces a unique value, the output of $f$ for the next element of the sequence.

\section*{Addition}
- Only for $N \to N$, obviously.

$$m \in N$$
$$0 + m := m$$

Assume inductively that $n+m, n \in N$ is defined. 

\begin{center}
	We now define $(n++) + m := (n+m)++$
	
	We can now recursively "count".
\end{center}

Note that $3+5$ involves counting 5 times, while $5+3$ involves counting 3 times.

\subsection*{Lemma 2.2.2 $\forall n \in \mathbb{N} n+0=n$}
We cannot just prove this from the definition of addition, as its in reverse.

\subsubsection*{Proof:}
We use induction.

Base case: $0+0=0$, we \textit{can} get this one by definition.

We now suppose inductively that $n+0=n$.

By definition. $n++~+0=(n+0)++$, which $=n++$ since we have assumed inductively that $(n+0)=n$.

This concludes the proof.


\subsection*{Lemma 2.2.3 $\forall n,m \in \mathbb{N}, n+(m++)=(n+m)++$}
\subsubsection*{Proof:}
We keep $m$ fixed. 

Consider $n=0$, $\to 0+(m++)=(0+m)++$. We know this is true by definition.

We assume what we are trying to prove is true.
Now we must show that $(n++)+(m++)=((n++)+m)++$.

By definition of addition we have that the left side is $(n+(m++))++$.
By the inductive hypothesis we have assumed that this $=((n+m)++)~~++$.

The right side then becomes $(n+m)++~++$, which $=((n+m)++)++$.
This concludes the induction.

\subsection*{Proposition 2.2.4 $\forall n, m \in \mathbb{N}, n+m = m+n$ (Conmutativity) }
\subsubsection*{Proof:}
We keep $m$ fixed. This lets us prove the base case ($0+m = m+0$) by definition. We now suppose that $n+m = m+n$ and have to prove $(n++) + m = m + (n++)$. 
\hspace{0.5cm}

\begin{tabular}{l p{13cm}}
	Left side  & 
	By definition of addition we get $(n++) + m = \colorbox{lightgray}{(n + m)++}$. 
	\\
	Right side & 
	By Lemna 2.2.3, we get $m + (n++) = (m + n)++$, but by our previous assumption, $n + m = m + n$ and $\therefore ((m + n)++ = \colorbox{lightgray}{(n + m)++} $
\end{tabular}

\subsection*{Proposition 2.2.5 $\forall a,b,c \in \mathbb{N}, (a + b) + c = a + (b + c)$ (Asociativity)}
\subsubsection*{Proof: Exercise 2.2.1}
\paragraph{Hint:}
Modify proofs of 2.2.2, 2.2.3 \& 2.2.4

something something inductive assumption $\to$ both sides $= (a + b + c) ++$

Let us induct on $a$.
\vspace{\parskip}

\paragraph{Base case: $a = 0$}~

\begin{center}
	\begin{tabular}{l l}
		$a = 0 \to$                & $(0 + b) + c = 0 + (b + c)$                     \\
		By definition of addition: & $\colorbox{lightgray}{b + c} = b + c ~~\square$ \\
	\end{tabular}
\end{center}

\paragraph{Induction:} Assume $(a + b) + c = a + (b + c)$

\begin{center}
	\begin{tabular}{l l}
		Prove:                             & $(a++~ + b) + c = a++~ + (b + c)$         \\
		By definition of addition:         & $(a + b)++~ + c = a++~ + (b + c)$         \\
		Same thing:                        & $(a + b)++~ + c = (a + (b + c))++$        \\
		One more time:                     & $((a + b) + c)++ = (a + (b + c))++$       \\
		We have already assumed asociativity. & $(a + b + c)++ = (a + b + c)++ ~~\square$
	\end{tabular}
\end{center}


\subsection*{Proposition 2.2.6 (Cancellation law)}
$\text{Let}~ a,b,c \in \mathbb{N} \ni a+b=a+c \text{. } \Rightarrow b=c$

\begin{itemize}
	\item We don't have subtraction
	\item But this does the job and also helps define the integers later
\end{itemize}

\subsubsection*{Proof}
We induct on $a$.

\paragraph{Base case: } $a = 0$
\begin{center}
	\begin{tabular}{l l}
		$a = 0 \to$                & $0 + b = 0 + c$ \\
		By definition of addition: & $\square$       \\
	\end{tabular}
\end{center}

\paragraph{Induction:} Assume $a + b = a + c \to b = c$
\begin{center}
	\begin{tabular}{l l}
		Prove:                     & $(a++) + b = (a++) + c$ \\
		By definition of addition: & $(a + b)++ = (a + c)++$ \\
		By Axiom 4:                & $(a + b) = (a + c)$     \\
		We assume cancelation      & $b = c ~~ \square$      \\
	\end{tabular}
\end{center}

\subsection*{Definition 2.2.7 (Positive natural numbers)}
$n \in \mathbb{N+} \iff n \neq 0$

\subsection*{Proposition 2.2.8}
$ a \in \mathbb{N+} ~\land~ b \in \mathbb{N} \to (a + b) \in \mathbb{N+}$

\subsubsection*{Proof:}
We induct on $b$.

\paragraph{Base case:} $b = 0$
\begin{center}
  \begin{tabular}{l l}
    $b = 0 \to$         & $a + 0 = a$ \\
    $a \in \mathbb{N+}$ & $\square$\\
  \end{tabular}
\end{center}

\paragraph{Induction:} Assume $ a \in \mathbb{N+} ~\land~ b \in \mathbb{N} \to (a + b) \in \mathbb{N+}$
\begin{center}
  \begin{tabular}{l l}
		Prove:              & $a + (b++) \in \mathbb{N+}$ \\
		By Lemna 2.2.3      & $a + (b++) = (a + b)++$    \\
		By Axiom 3:         & $(a + b)++ \neq 0 ~~\square$
  \end{tabular}
\end{center}

\subsection*{Corollary 2.2.9}
$ \{a,b\} \in \mathbb{N} \land a + b = 0 \to a = 0 \land b = 0$

\subsubsection*{Proof}
If $a$ or $b$ $\neq 0$ (are positive), by Proposition 2.2.8 (and conmutativity) $a + b \in \mathbb{N+}$. $\therefore$, by contradiction, $a,b = 0$. 


\subsection*{Lemma 2.2.10}
$a \in \mathbb{N+} \to \exists |~ b \in \mathbb{N} ~\text{s.t.}~ (b++) = a$

\subsubsection*{Proof: Exercise 2.2.2}
\paragraph{Hint:} Use indiction. (The induction hypotesis is not actually used, but that doesn't undermine the validity of the argument)

\vspace{\parskip}
We induct on $b$.

\paragraph{Base case:} $b = 0$
\begin{center}
  \begin{tabular}{l l}
		$b = 0 \to$        & $0++ = 1 \to a = 1$ \\ 
		By Axiom 3:				 & $a \in \mathbb{N+} ~~\square$
  \end{tabular}
\end{center}

\paragraph{Induction:} Assume $a \in \mathbb{N+}$
\begin{center}
  \begin{tabular}{l l}
		Prove:             & $(b++)++ = a \to a \in \mathbb{N+}$ \\ 
		By Axiom 3:        & $(b++), (b++)++ \neq 0 \to (b++)++ \in \mathbb{N+} \to a \in \mathbb{N+} ~~\square??????$
  \end{tabular}
\end{center}

\paragraph{Definition 2.2.1: Ordering of $\mathbb{N}$}~

$n \geq m \iff \exists \colorbox{lightgray}{$a$}~ s.t.~ n = m + a$

$n > m \iff n \geq m \land n \neq m$

\paragraph{Proposition 2.2.12: (Basic properties of the natural nubers)}

{
	\renewcommand{\theenumi}{(\alph{enumi})}
	\begin{enumerate}
		\item $x \geq x$
			\begin{center}
				\begin{tabular}{l l}
					Def. Addition: & $x + 0 = x$ \\
					               & $\therefore \colorbox{lightgray}{$a$} = 0 \to x \geq x$
				\end{tabular}
			\end{center}

		\item $a \geq b \land b \geq c \to a \geq c$
			\begin{center}
				\begin{tabular}{l l}
					$a = b + n_1; b = c + n_2$ \\
					$a = c + n_1 + n_2$
				\end{tabular}
			\end{center}
			
		\item $a \geq b \land b \geq a \to a = b$
			\begin{center}
				\begin{tabular}{l l}
					                        & $a = b + n_1;~ b = a + n_2$ \\
					                        & $a = a + n_2 + n_1$ \\
					Asso. \& Def. Addition: & $a + 0 = a + (n_2 + n_1)$ \\
					Cancelation law:        & $(n_2 + n_1) = 0$ \\
					Collary 2.2.9:          & $n_1, n_2 = 0$ \\
					Def. Addition:          & $a = b + 0 \to a = b$
				\end{tabular}
			\end{center}
			
		\item $a \geq b \iff a + c \geq b + c$
			\begin{center}
				\begin{tabular}{l l}
					%$a \geq b \to$         & $a = b + n_1$ \\
					$a + c \geq b + c \to$ & $a + c = b + c + n_1$ \\
					Cancellation law:      & $a = b + n_1 \therefore a \geq b ???$
				\end{tabular}
			\end{center} 
			
		\item $a < b \iff a++ \leq b$
			\begin{center}
				\begin{tabular}{l l}
                         & $a++ \geq b \to a++~ + n_1 = b$ \\
          Def. Addition: & $a++ = a + 1$ \\
          							 & $n_2 = n_1 + 1 \to a + n_2 = b$ \\
          Def $ > $ 1    & $n_2 > 0$ \\
          							 & $a < b$
				\end{tabular}
			\end{center}
			\\~\\

		\item $a < b \iff b = a + d \land d \in \mathbb{N}+$
			\begin{center}
				\begin{tabular}{l l}
					Proposition 2.2.12 e: & $a++ \leq b$ \\
					Def. Addition         & $a + 1 \leq b$ \\
					Def. $ < $            & $a + 1 + n_1 = b$ \\
					Def. Addition         & $a + (n_1++) = b$ \\
					Axiom 2.3             & $(n_1++) \neq 0$ \\
					Def. Positive         & $n_1 \text{ is positive}$
				\end{tabular}
			\end{center}
	\end{enumerate}
}

\paragraph{Proposition 2.2.13: Trichotomy of order for natural numbers}~

$a,b \in \mathbb{N} \to (a < b \lor a = b \lor a > b)$
\begin{enumerate}
	\item $a = 0 \to a \leq b~ \forall b$ (why?) Induction.
			\begin{center}
				\begin{tabular}{l l}
					$b = 0$ & $0 + 0 = b$ \\
					$b ++ $ & $0 + n_1 = b++$ \\
									& $\forall b++, n_1 = b++$~~~  or something like that
				\end{tabular}
			\end{center}

	\item $a > b \to a++ > b$ (why?)
			\begin{center}
				\begin{tabular}{l l}
					Proposition 2.2.12 f: & $a = b + n_1, n_1 \text{ is positive}$ \\
					Axiom 2.4:            & $a++ = (b + n_1)++$ \\
					Associativiy          & $a++ = b + (n_1++)$ \\
					Axiom 2.3:            & $n_1++ \text{ is positive}$ \\
					Proposition 2.2.12 f: & $a++ > b$
				\end{tabular}
			\end{center}

	\item $a = b \to a++ > b$ (why?)
			\begin{center}
				\begin{tabular}{l l}
					Def. Addition: & $a++ = b++ = a + 1$
				\end{tabular}
			\end{center}
\end{enumerate}

\paragraph{Proposition 2.2.14: Strong principle of induction}

Let $m_0$ be some natural number and $P(m)$ be a property of an arbetrary natural number $m$. Suppose that $\forall m \geq m_0:~ P(m')~ \forall~ m_0 \leq m' < m \to P(m)$. It can then be concluded that $P(m)~ \forall~ m \geq m_0$.

\paragraph{Proof:}
Let $Q(n)$ be the property that $P(m)~ \forall~ m_0 \leq m < n$. The hypothesis can now be re-written as: $Q(n) \to P(n)$.

\begin{center}
	\begin{tabular}{l l}
		\textbf{Base case: $n = 0$} \\
		Im not proving this:   & $\nexists m \text{ s.t. } m_0 \leq m < 0$ \\
		\\
		\textbf{Assume $Q(n) \to P(n)$. Prove $Q(n++)$} \\
		$Q(n) \land P(n) \to$  & $P(m)~ \forall~ m_0 \leq m \colorbox{pink}{$\leq$} n$ \\
		Proposition 2.2.12 e:  & $P(m)~ \forall~ m_0 \leq m < n++$ \\
		Def. $Q(n)$            & $Q(n++)~~ \square$
	\end{tabular}
\end{center}


\end{document}
