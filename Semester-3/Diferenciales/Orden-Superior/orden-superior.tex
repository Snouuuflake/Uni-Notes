%! TeX program = lualatex
%tags! segundo orden orden superior
\documentclass[letterpaper]{article}
\usepackage[margin=1in]{geometry}
\usepackage{amsmath}
\usepackage{amssymb}
\usepackage{circuitikz}
\usepackage[no-math]{fontspec}
\usepackage[]{gruvboxpalette}
\usepackage{hyperref}
\usepackage{newtxsf}
\usepackage[explicit]{titlesec}
\usepackage{tikz}
\usepackage{varwidth}
\usetikzlibrary{calc}
\usetikzlibrary{positioning}
\usetikzlibrary{arrows.meta}
\usepackage[most]{tcolorbox}
\usepackage{tabularray}
\DefTblrTemplate{firsthead, middlehead,lasthead}{default}{}
\DefTblrTemplate{capcont}{default}{}
\DefTblrTemplate{contfoot-text}{normal}{} \SetTblrTemplate{contfoot-text}{normal} \DefTblrTemplate{conthead-text}{normal}{} \SetTblrTemplate{conthead-text}{normal}
\UseTblrLibrary{counter}
\hypersetup{
  colorlinks  = true,
  urlcolor    = blue,
  linkcolor   = font,
  citecolor   = font
}
\usepackage[most]{tcolorbox}

\setmainfont{NotoSans-Regular}[
Path           = /home/snouflake/.fonts/ ,
Extension      = .ttf ,
BoldFont       = NotoSans-Bold ,
ItalicFont     = NotoSans-Italic ,
BoldItalicFont = NotoSans-BoldItalic,
] 

\newtcolorbox{defbox}[3][]{%
  colback=blue!30!background,
  coltitle=blue!15!black,
  coltext=font,
  title filled=false,
	enhanced,
  detach title,
  tile,
  before upper={\tcbtitle\medskip\\},
  borderline west={2mm}{0pt}{blue},
  % attach boxed title to top center={yshift=-2mm},
  leftrule=2mm,
  toprule=0mm,
  bottomrule=0mm,
  rightrule=0mm,
  arc=0mm,
	title={Definición:~#2},
	#1
}

\usepackage[italic]{mathastext}

\setlength\parindent{0pt}

\newcommand{\linesection}[1]{\bigskip\textbf{#1}{\hspace{10pt}\leavevmode\leaders\hrule height0.42em depth-0.3em\hfill\kern0pt}\bigskip}

\def \T{Ecuaciones Diferenciales}
\def \S{Ecuaciones de orden superior}

\begin{document}
\begin{tikzpicture}[inner sep=0pt,color=font]
  \node[anchor=west,align=left,line width=0pt] 
    (title) at (0,0) {\huge\bfseries\noindent\T\\\smallskip\Large\bfseries\S}
    ;
\end{tikzpicture}

% \textbf{Ecuaciones lineales homogéneas}{\hspace{10pt}\leavevmode\leaders\hrule height0.42em depth-0.3em\hfill\kern0pt}

\makeatletter
\renewcommand\tagform@[1]{\maketag@@@{\ignorespaces\bfseries#1\unskip\@@italiccorr}}
\makeatother

\begin{longtblr}{
    colspec={@{}Q[4cm,cmd=\textbf,h] >{\begin{minipage}{\linewidth}}X<{\end{minipage}}@{}},
    rowsep={7pt},
  }
  Ecuaciones lineales homógeneas
  &
  \begin{equation}
    \text{Sea } a y' + by = 0 ~s.t.~ a \neq b 
    \label{eq:deg1init}
  \end{equation}
  \begin{flalign}
    & \Rightarrow y' = ky\\
    & \therefore y = e^{mx} \land y' = m e^{mx}.
    \label{eq:1degysol}
    \\
    & \therefore a me^{mx} + b e^{mx} = 0 \therefore e^{mx} (am + b) = 0 \label{eq:1degsub}
  \end{flalign}
  *donde $m$ es una solución a $am + b = 0$ porque, factorizando \ref{eq:1degsub}:2, $e^{mx} nunca puede ser 0$.
  \medskip

  Parecería que para la solución particular $y = e^{\frac{-5 x}{2}}$ , la solución general es $y = c_1 e^{\frac{-5 x}{2}}$, lo cual, derivando, tiene sentido, supongo.
  \\
  De segundo grado
  &
  Consideremos:

  \begin{equation}
    a y'' + b y' + cy = 0
  \end{equation}

  Aplicando lo el mismo proceso que en la ec. \ref{eq:1degysol}, se obtiene.
  
  \begin{equation}
    a m^2 e^{mx} + b m e^{mx} + c e^{mx} = 0 \land e^{mx} \left(a m^2 + b m + c\right) = 0
  \end{equation}

  De la misma forma, $m$ debe ser solución de:
  \begin{equation}
    a m^2 + bm + c = 0
    \label{eq:2degquad}
  \end{equation}
  
  De la ec. \ref{eq:2degquad} se obtienen $m_1$ y $m_2$, por lo cual se producen tres casos:
  
\\&
  \linesection{Caso 1: Raíces reales y distintas}

  Se obtienen dos soluciones linealmente independientes sobre $\mathbb{R}$ de la forma $y = e^{m_n x}$ por lo cual se obtiene como solución general

  \begin{equation}
    y = c_1 e^{m_1 x} + c_2 e^{m_2 x}
    \label{eq:deg2case1sol}
  \end{equation}

\\&
  \linesection{Caso 2: Raíces reales repetidas}

  Se obtienen dos soluciónes sabe Dios cómo:

  \begin{equation}
    y = c_1 e^{m_1 x} + c_2 x e^{m_1 x}
    \label{eq:deg2case2sol}
  \end{equation}

\\&
  \linesection{Caso 3: Raíces complejas conjugadas}

  Siendo $m_1$ y $m_2$ complejas, se pueden escribir como $\alpha + i \beta$ y $\alpha - i \beta$ donde $\alpha, \beta \in \mathbb{R} \,\land > 0$.
  \medskip

  De aquí podríamos tener como solución:
  \begin{equation}
    y = C_1 e^{m_1 x} + C_2 e^{m_2 x}
    \label{eq:deg3weaksol}
  \end{equation}

  pero la vida no es tan bonita.
  \medskip

  Con la fórmula de Euler:
  \begin{equation}
    e^{i \theta} = \cos(\theta) + i \sin(\theta), \theta \in \mathbb{R}
    \label{eq:formulaeuler}
  \end{equation}


  y usando $\sin(-x) = -\sin(x)$ ~y~ $\cos(-x) = \cos(x)$, se obtiene que:
  \begin{equation}
    e^{i \beta x} = \cos(\beta x) + i \sin(\beta x)
    \text{ y }
    e^{-i \beta x} = \cos(\beta x) - i \sin(\beta x)
    \label{eq:deg2eandminuse}
  \end{equation}
  \medskip

  Para obtener la solución general (sabe Dios por qué), se consideran los dos sigueintes casos:

    \medskip
    \textbf{Caso I: $C_1 = C_2 = 1$}
    \medskip
    \begin{align}
      y_1 = e^{\alpha x} \left(e^{i \beta x} + e^{-i \beta x}\right) = 2 e^{\alpha x} \cos(\beta x)
    \end{align}

    \medskip
    \textbf{Caso II: $C_1 =1, C_2 = -1$}
    \medskip
    \begin{align}
      y_2 = e^{\alpha x} \left(e^{i \beta x} - e^{-i \beta x}\right) = 2 i e^{\alpha x} \sin(\beta x)
    \end{align}

    Dado que en algún teorema en algún lugar dice que si $y_1$ es solución a una ecuación lineal homogénea, entonces $c_1 y_1$ también es solución (en este caso, $c_1 = -i$ para $y_2$), \textit{$e^{\alpha x} \cos(\beta x)$ y $e^{\alpha x} \sin(\beta x)$ son soluciónes reales}.
    \medskip

    Como estas soluciones forman un conjunto fundamental sobre $\mathbb{R}$, la solución general es:

    \begin{equation}
      y = e^{\alpha x} \left(
        c_1 \cos(\beta x) + c_2 \sin(\beta x)\right
        )
      \label{eq:deg2case3gensol}
    \end{equation}
\\&
  \linesection{Valores reales}

  Para $y(m) = n$ y $y'(j) = k$, primer se substituye $m$ y $n$ y se despeja $c_1$. Luego, se deriva, se substituye $j$ y $k$ y se despeja $c_2$.
  \\
  Orden $n$
  &
  Para una ecuación de orden $n$, con coeficientes $a_{n\to0}$, se resuelve el siguiente polinomio de grado $n$:
  \begin{equation}
    a_n m^n +
    a_{n-1} m^{n-1} +
    \dots +
    a_{1} m^{1} +
    a_{0}
    = 0
    \label{eq:degnmpoly}
  \end{equation}

  Si todas las racies de ec. \ref{eq:degnmpoly} son reales, entonces la solución general es
  \begin{equation}
    y = 
    c_{1} e^{m_1 x} +
    c_{2} e^{m_2 x} +
    \dots
    c_{n} e^{m_n x} +
    \label{eq:degncase1sol}
  \end{equation}

  \\
  Reducción de órden
  &
  Sea:
  \begin{equation}
    a_2\left(x\right) y'' + a_1\left(x\right) y' + a_0\left(x\right) y = 0
    \label{eq:reddeg2init}
  \end{equation}
  Se puede obtener
  \begin{equation}
    y'' + 
    \frac{a_1\left(x\right)}{a_2\left(x\right)} y' + 
    \frac{a_0\left(x\right)}{a_2\left(x\right)} y = 
    y'' + P\left(x\right) y' + Q\left(x\right) y = 0
    \label{eq:reddeg2init2}
  \end{equation}
  \medskip
  
  Se conoce una solución particular $y_1\left(x\right)$.
  \medskip
  
  Sen nesecita una segunda solución particular $y_2\left(x\right)$.
  \medskip

  Se propone:
  \begin{equation}
    y_2\left(x\right) = u\left(x\right) y_1\left(x\right)
    \label{eq:reddeg2y2proposal}
  \end{equation}
  
  \begin{center}
    $\dots$
  \end{center}

  \begin{equation}
    u = \int \frac{e^{- \int P(x) \,dx}}{y_1^2} \,dx
    \label{eq:reddeg2y2sol}
  \end{equation}

  \begin{equation}
    y(x) = c_1 y_1 + c_2 u y_1
    \label{eq:reddeg2sol}
  \end{equation}




\end{longtblr}
\end{document}




