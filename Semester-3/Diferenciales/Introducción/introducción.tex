%! TeX program = lualatex
%tags! into lineal ode edo
\documentclass[letterpaper]{article}
\usepackage[margin=1in]{geometry}
\usepackage{amsmath}
\usepackage{amssymb}
\usepackage[no-math]{fontspec}
\usepackage[bg,fg]{gruvboxpalette}
\usepackage{hyperref}
\usepackage{newtxsf}
\usepackage[explicit]{titlesec}
\usepackage{tikz}
\usetikzlibrary{calc}
\usetikzlibrary{positioning}
\usetikzlibrary{arrows.meta}
\usepackage[most]{tcolorbox}
\usepackage{tabularray}
\DefTblrTemplate{firsthead, middlehead,lasthead}{default}{}
\DefTblrTemplate{capcont}{default}{}
\DefTblrTemplate{contfoot-text}{normal}{} \SetTblrTemplate{contfoot-text}{normal} \DefTblrTemplate{conthead-text}{normal}{} \SetTblrTemplate{conthead-text}{normal}
\UseTblrLibrary{counter}
\hypersetup{
  colorlinks  = true,
  urlcolor    = Blue,
  linkcolor   = Blue,
  citecolor   = Blue
}
\usepackage[most]{tcolorbox}

\setmainfont{NotoSans-Regular}[
Path           = /home/snouflake/.fonts/ ,
Extension      = .ttf ,
BoldFont       = NotoSans-Bold ,
ItalicFont     = NotoSans-Italic ,
BoldItalicFont = NotoSans-BoldItalic,
] 

\newtcolorbox{defbox}[2][]{%
  colback=blue!30!background,
  coltitle=blue!15!black,
  coltext=font,
  title filled=false,
	enhanced,
  detach title,
  tile,
  before upper={\tcbtitle\medskip\\},
  borderline west={2mm}{0pt}{blue},
  % attach boxed title to top center={yshift=-2mm},
  leftrule=2mm,
  toprule=0mm,
  bottomrule=0mm,
  rightrule=0mm,
  arc=0mm,
	title={Definición:~#2},
	#1
}

\setlength\parindent{0pt}

\usepackage{mathastext}

\def \T{Programación de Modelos en\\Ecuaciones Diferenciales}
\def \S{2025-01-13}

\begin{document}
\begin{tikzpicture}[inner sep=0pt,color=font]
  \node[anchor=west,align=left,text width=\linewidth] 
    (title) at (0,0.9) {\Huge\bfseries\T};
  \node[anchor=west,align=left] 
    (subtitle) at (0,-0.5) {\Large\bfseries\S};
\end{tikzpicture}
\vspace{16pt}

\begin{defbox}{Ecuación diferencial (ED)}
  Una ecuación que contiene las derivadas de una o más funciones desconicidas (o variables dependientes)
  \begin{center}
    \[
      \frac{dy}{dx} = 2x 
      \to
      \int dy ~dx = \int 2xdx
    \]
    \[
      \text{Conjunto de solución general: } y = 2x^{2} + C
    \]
  \end{center}
\end{defbox}

\begin{longtblr}{
    colspec={@{}Q[h,4cm,cmd=\textbf] X[h]@{}},
    rowsep={7pt},
  }

  Ecuación diferencial ordinaria 
  & Ecuación diferencial que contiene derivadas totales / ordinarias
  \\
  Ecuación diferencial parcial
  & Ecuación diferencial que contiene derivadas parciales
  \\
  Variables dependientes
  & {
    Dada la ecuación diferencial $(x^2 + y^2) ~dx + xy ~dy=0$, si se divide toda entre $dx$, resulta $(x^2 + y^2) + xy \frac{dy}{dx} = 0$, y la variable dependiente es $y$. Si, en cambio, se divide entre $dy$, resulta $(x^2 + y^2) \frac{dx}{dy} + xy = 0$, y la variable dependiente es $x$.
  }
  \\
  Orden
  & {
    El orden de una ecuación diferencial es el orden de su derivada mayor. 
    Forma general de una ecuación diferencial ordinaria de orden $n$: $f(x,y,y',\dots,y^{\left(n\right)}) = 0$, donde $y^{\left(n\right)} = \frac{d^n y}{dx^n}$.
  }
  \\
  Grado
  & \begin{minipage}{\linewidth}
    El grado algebráico de la derivada mayor.

    \begin{center}
      \begin{tblr}{%
          colspec={l l}
        }
        $y'' + y' + y = \sin(x)$ & orden 2, grado 1 \\
        $3y'' + y'^2 = 0$ & orden 2, grado 1 \\
        $y''^2 + 2 y' = 0$ & orden 2, grado 2 \\
      \end{tblr}
    \end{center}

  \end{minipage}
  \\
  EDO Lineal
  & {
    $
    a_n(x) \frac{d^n y}{dx^n} + 
    a_{n-1}(x) \frac{d^{n-1} y}{dx^{n-1}} + 
    \dots + 
    a_1(x) \frac{d y}{dx} + 
    a_0(x) \frac{d y}{dx}
    = g(x)
    $

    *El exponente de cada derivada es $1$ y el lado derecho es únicamente función de $x$.
  }
  \\
  Solución general de la EDO
  & \begin{minipage}{\linewidth}
    Dado $f(x,y' , \dots,y^{\left(n\right)}) = 0$, $y = y(x, C_1, C_2,\dots,C_n)$.
    \medskip
    
    \begin{tblr}{colspec={@{}X[m] X[m]}}
      Cada vez que se integra, se disminuye el grado de la ecuación por $1$, y se obtiene una nueva constante de integración. 
    & \begin{minipage}{\linewidth}
      $$\begin{aligned}
        \frac{ d^2y }{ dx^2 } &= 3x \\
        \frac{dy}{dx} &= \frac{3}{2} x^2 + C_1 \\
        y &= \frac{3}{6} x^3 + C_1 x + C_2 \\
      \end{aligned}$$
    \end{minipage}
    \end{tblr}

    Para obtener una solución particular para una EDO de orden $n$, una solución con todas las constantes remplazadas por escalares, de acuerdo a los requisitos de un problema real, se requieren de $n$ condiciónes iniciales; el valor de las primeras $n$ derivadas de $y$ para algún punto $x_0$.

    \begin{tblr}{colspec={c X[m]}}
      Se obtiene un sistema lineal de $n$ ecuaciones:
      & \begin{minipage}{\linewidth}
        $$
          \begin{cases}
            y'(x_0)  = y_0' \\
            y''(x_0) = y_0'' \\
            \dots  \\
            y^{\left(n\right)}(x_0)  = y_0^{\left(n\right)} \\
          \end{cases}
        $$
      \end{minipage}
    \end{tblr}
  \end{minipage}

\end{longtblr}

\end{document}
