%! TeX program = lualatex
%tags! variables separables vairables
\documentclass[letterpaper]{article}
\usepackage[margin=1in]{geometry}
\usepackage{amsmath}
\usepackage{amssymb}
\usepackage[no-math]{fontspec}
\usepackage[bg,fg]{gruvboxpalette}
\usepackage{hyperref}
\usepackage{newtxsf}
\usepackage[explicit]{titlesec}
\usepackage{tikz}
\usetikzlibrary{calc}
\usetikzlibrary{positioning}
\usetikzlibrary{arrows.meta}
\usepackage[most]{tcolorbox}
\usepackage{tabularray}
\DefTblrTemplate{firsthead, middlehead,lasthead}{default}{}
\DefTblrTemplate{capcont}{default}{}
\DefTblrTemplate{contfoot-text}{normal}{} \SetTblrTemplate{contfoot-text}{normal} \DefTblrTemplate{conthead-text}{normal}{} \SetTblrTemplate{conthead-text}{normal}
\UseTblrLibrary{counter}
\hypersetup{
  colorlinks  = true,
  urlcolor    = Blue,
  linkcolor   = Blue,
  citecolor   = Blue
}
\usepackage[most]{tcolorbox}

\setmainfont{NotoSans-Regular}[
Path           = /home/snouflake/.fonts/ ,
Extension      = .ttf ,
BoldFont       = NotoSans-Bold ,
ItalicFont     = NotoSans-Italic ,
BoldItalicFont = NotoSans-BoldItalic,
] 

\newtcolorbox{defbox}[2][]{%
  colback=blue!30!background,
  coltitle=blue!15!black,
  coltext=font,
  title filled=false,
	enhanced,
  detach title,
  tile,
  before upper={\tcbtitle\medskip\\},
  borderline west={2mm}{0pt}{blue},
  % attach boxed title to top center={yshift=-2mm},
  leftrule=2mm,
  toprule=0mm,
  bottomrule=0mm,
  rightrule=0mm,
  arc=0mm,
	title={Definición:~#2},
	#1
}

\setlength\parindent{0pt}

\usepackage[italic]{mathastext}

\def \T{Programación de Modelos en\\Ecuaciones Diferenciales}
\def \S{Ecuaciones Lineales}

\begin{document}
\begin{tikzpicture}[inner sep=0pt,color=font]
  \node[anchor=west,align=left,text width=\linewidth] 
    (title) at (0,0.9) {\Huge\bfseries\T};
  \node[anchor=west,align=left] 
    (subtitle) at (0,-0.5) {\Large\bfseries\S};
\end{tikzpicture}
\vspace{16pt}


\begin{longtblr}{
    colspec={@{}Q[h,4cm,cmd=\textbf] X[t]@{}},
    rowsep={7pt},
  }
  Ecuación diferencial de primer orden 
  & \begin{minipage}{\linewidth}
    \[
      a_1(x) \frac{dy}{dx} + a_0(x) y = g(x)
    \]

    Forma estándar:

      \begin{align*}
        \frac{1}{a_1(x)}\left(a_1(x) \frac{dy}{dx} + a_0(x) y\right) &= \frac{1}{a_1(x)}{g(x)}\\
        \frac{dy}{dx} + P(x) y &= R(x)
      \end{align*}
  \end{minipage}
  \\
  Factor Integrante
  & \begin{minipage}{\linewidth}
    Imagina una función $u(x)$ tal que $\frac{d}{dx} \left(u(x) y\right) = u(x) R(x)$.
    \medskip

    Multiplicando la forma estándar por $u(x)$, se obtiene: 

    \[
     \colorbox{blue!50!background}{$\displaystyle u(x) \frac{dy}{dx} + u(x) P(x) y$} = u(x) R(x)
    \]

    Expandiendo la derivada de $u(x) y$, se obtiene:
    \[
      \frac{d}{dx} \left(u(x) y\right) = 
      u(x) \frac{dy}{dx} + \frac{du}{dx} y = \colorbox{blue!50!background}{$\displaystyle u(x) \frac{dy}{dx} + u(x) P(x) y$}
    \]

    $\therefore$ $\frac{du}{dx} y = u(x) P(x) y$ \& $\frac{du}{dx} = u(x) P(x)$.
    \bigskip

    \begin{align*}
      \Rightarrow \frac{1}{u} ~du &= P(x) ~dx \\
      \int \frac{1}{u} ~du &= \int P(x) ~dx \\
      \ln(u) &=  \int P(x) ~dx + C \\
      u &= e^{\int P(x) ~dx + C} \\
      u &= e^{\int P(x) ~dx} e^{C} \\
      u &= e^{\int P(x) ~dx} C \\
      C = 1 \to u &= e^{\int P(x) ~dx}\\
      \\
      \therefore \frac{d}{dx} \left[e^{\int P(x) ~dx} y\right] &= e^{P(x) ~dx} R(x)
    \end{align*}

  \end{minipage}


\end{longtblr}

\end{document}


