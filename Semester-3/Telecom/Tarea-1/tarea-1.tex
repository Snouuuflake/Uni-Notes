%! TeX program = lualatex
%tags! intro first class telecom
\documentclass[letterpaper]{article}
\usepackage[margin=1in]{geometry}
\usepackage{amsmath}
\usepackage{amssymb}
\usepackage[no-math]{fontspec}
\usepackage{gruvboxpalette}
\usepackage{hyperref}
\usepackage{newtxsf}
\usepackage[explicit]{titlesec}
\usepackage{tikz}
\usetikzlibrary{calc}
\usetikzlibrary{positioning}
\usetikzlibrary{arrows.meta}
\usepackage[most]{tcolorbox}
\usepackage{tabularray}
\DefTblrTemplate{firsthead, middlehead,lasthead}{default}{}
\DefTblrTemplate{capcont}{default}{}
\DefTblrTemplate{contfoot-text}{normal}{} \SetTblrTemplate{contfoot-text}{normal} \DefTblrTemplate{conthead-text}{normal}{} \SetTblrTemplate{conthead-text}{normal}
\UseTblrLibrary{counter}
\hypersetup{
  colorlinks  = true,
  urlcolor    = Blue,
  linkcolor   = Blue,
  citecolor   = Blue
}
\usepackage[most]{tcolorbox}

\setmainfont{NotoSans-Regular}[
Path           = /home/snouflake/.fonts/ ,
Extension      = .ttf ,
BoldFont       = NotoSans-Bold ,
ItalicFont     = NotoSans-Italic ,
BoldItalicFont = NotoSans-BoldItalic,
] 

\newtcolorbox{defbox}[3][]{%
  colback=blue!30!background,
  coltitle=blue!15!black,
  coltext=font,
  title filled=false,
	enhanced,
  detach title,
  tile,
  before upper={\tcbtitle\medskip\\},
  borderline west={2mm}{0pt}{blue},
  % attach boxed title to top center={yshift=-2mm},
  leftrule=2mm,
  toprule=0mm,
  bottomrule=0mm,
  rightrule=0mm,
  arc=0mm,
	title={Definición:~#2},
	#1
}

\setlength\parindent{0pt}

\usepackage{mathastext}

\def \T{Introducción a las Telecomunicaciones}
\def \S{Espectro Radio-eléctrico}

\begin{document}
\begin{tikzpicture}[inner sep=2pt,color=font]
  \node[anchor=west,align=left] 
    (title) at (0,0) {\huge\bfseries\T\\\Large\bfseries\S}
    ;
  \node[anchor=east]
    {
      \begin{tikzpicture}[inner sep=0pt,draw=font]
        \coordinate (legLO) at ($(-2pt,0) + (250:22pt)$);
        \coordinate (legLI) at ($(legLO) + (2pt,0)$);
        \coordinate (legRO) at ($(2pt,0) + (-70:22pt)$);
        \coordinate (legRI) at ($(legRO) + (-2pt,0)$);
        \draw[fill=font]
          (legLI) to[out=45,in=135] (legRI) -- (legRO) to[out=125,in=55] (legLO) -- cycle
          ;
        \draw[fill=font]
          (-2pt,0) -- (legLO) -- (legLI) -- ++(70:18pt) -- cycle;
        \draw[fill=font]
          (2pt,0) --  (legRO) -- (legRI) -- ++(-250:18pt) -- cycle;
          ;
        \draw[fill=font] 
          (4pt,0) arc (0:360:4pt); 
        \draw[fill=font]
          (-7pt,-5pt) to[out=135,in=225] ++(0,10pt) -- ++(-2pt,0) to[out=225,in=135] ++(0,-10pt) -- cycle 
          ;
        \draw[fill=font]
          (-12pt,-8pt) to[out=135,in=225] ++(0,16pt) -- ++(-2pt,0) to[out=225,in=135] ++(0,-16pt) -- cycle
          ;
        \draw[fill=font]
          (7pt,-5pt) to[out=45,in=-45] ++(0,10pt) -- ++(2pt,0) to[out=-45,in=45] ++(0,-10pt)
          -- cycle
          ;
        \draw[fill=font]
          (12pt,-8pt) to[out=45,in=-45] ++(0,16pt) -- ++(2pt,0) to[out=-45,in=45] ++(0,-16pt)
          -- cycle
          ;
      \end{tikzpicture}
    }
    ;
\end{tikzpicture}

\section*{Tipos de comunicación}

\vspace{-2.5cm}

\begin{longtblr}{
    colspec={@{}Q[3cm,cmd=\textbf,h] X@{}},
    rowsep={13pt}
  }
  \\
  LF
  & \begin{minipage}{\linewidth}
    Freq: 30-300kHz
    \medskip

    Radio AM
    \medskip

    Banda kilometro
    \medskip

    Se pueden propagar reflejándose por la tierra o por la ionosfera
		
	\end{minipage}
  \\
  MF
  & \begin{minipage}{\linewidth}
    Freq: 300kHz-3MHz
    \medskip
    
    Radio AM
    \medskip

    Se pueden propagar reflejándose por la tierra o por la ionosfera
	\end{minipage}
  \\
  HF
  & \begin{minipage}{\linewidth}
    Freq: 3-30MHz
    \medskip
    
    Comunicación de larga distancia mediante la ionosfera
    \medskip

    Usos: militar, aviación, marítimo

	\end{minipage}
  \\
  VHF
  & \begin{minipage}{\linewidth}
    Freq: 30-300MHz
    \medskip

    Propagación por línea de vista
    \medskip

    Usos: WIFI, radio FM, audio digital, televisión (analógica)
	\end{minipage}
  \\
  UHF
  & \begin{minipage}{\linewidth}
    Freq: 300MHz-3GHz
    \medskip
    
    Propagación: Línea de vista. Permite reusar bandas
    \medskip

    Usos: televisón vía terrestre; celulares, 3 \& 4G; satélite, GPS, Wifi, Bluetooth;
	\end{minipage}
  \\
  SHF
  & \begin{minipage}{\linewidth}
    Freq: 3-30GHz (microondas)
    \medskip

    La longitud de onda permita que se transmitan de manera muy dirigida, por lo cual se usan para comunicación de punto a punto y radar.
    \medskip

    Usos: LAN WIFI, satélite
        
	\end{minipage}
  \\
  EHF
  & \begin{minipage}{\linewidth}
    Freq: 30-300GHz
    \medskip

    Propagación: Línea de vista, díficilmente pasan por paredes o plantas

    \medskip
    Usos: Radar de control de trio, scanners de seguridad en airopuertos, redes inalámbricas de rango corto
	\end{minipage}

\end{longtblr}
\end{document}
