%! TeX program = lualatex
%tags! energia fisica cinetica 
\documentclass[letterpaper]{article}
\usepackage[margin=1in]{geometry}
\usepackage{amsmath}
\usepackage{amssymb}
\usepackage{circuitikz}
\usepackage[no-math]{fontspec}
\usepackage[fg]{gruvboxpalette}
\usepackage{hyperref}
\usepackage{newtxsf}
\usepackage[explicit]{titlesec}
\usepackage{tikz}
\usepackage{varwidth}
\usetikzlibrary{calc}
\usetikzlibrary{positioning}
\usetikzlibrary{arrows.meta}
\usepackage[most]{tcolorbox}
\usepackage{tabularray}
\DefTblrTemplate{firsthead, middlehead,lasthead}{default}{}
\DefTblrTemplate{capcont}{default}{}
\DefTblrTemplate{contfoot-text}{normal}{} \SetTblrTemplate{contfoot-text}{normal} \DefTblrTemplate{conthead-text}{normal}{} \SetTblrTemplate{conthead-text}{normal}
\UseTblrLibrary{counter}
\hypersetup{
  colorlinks  = true,
  urlcolor    = Blue,
  linkcolor   = Blue,
  citecolor   = Blue
}
\usepackage[most]{tcolorbox}

\setmainfont{NotoSans-Regular}[
Path           = /home/snouflake/.fonts/ ,
Extension      = .ttf ,
BoldFont       = NotoSans-Bold ,
ItalicFont     = NotoSans-Italic ,
BoldItalicFont = NotoSans-BoldItalic,
] 

\newtcolorbox{defbox}[3][]{%
  colback=blue!30!background,
  coltitle=blue!15!black,
  coltext=font,
  title filled=false,
	enhanced,
  detach title,
  tile,
  before upper={\tcbtitle\medskip\\},
  borderline west={2mm}{0pt}{blue},
  % attach boxed title to top center={yshift=-2mm},
  leftrule=2mm,
  toprule=0mm,
  bottomrule=0mm,
  rightrule=0mm,
  arc=0mm,
	title={Definición:~#2},
	#1
}

\setlength\parindent{0pt}

\usepackage[italic]{mathastext}

\def \T{Física 1}
\def \S{Energía}

\begin{document}
\begin{tikzpicture}[inner sep=0pt,color=font]
  \node[anchor=west,align=left,line width=0pt] 
    (title) at (0,0) {\huge\bfseries\noindent\T\\\smallskip\Large\bfseries\S}
    ;
\end{tikzpicture}


\begin{longtblr}{
    colspec={@{}Q[4cm,cmd=\textbf,h] >{\begin{minipage}{\linewidth}}X<{\end{minipage}}@{}},
    rowsep={7pt},
  }
  Energía cinética
  &
  \[
    E_c = K = \frac{1}{2}\, m \left|v\right|^2 \hspace{10pt} \text{J\,[oules]}
  \]
  * Normalmente $\left|v\right|$, la rapidez, se escribe como $v$, pero tanto $E_c$, y $v$ en este caso, son escalares.

  \\
  Energía potencial gravitacional
  &
  \begin{equation}
    E_p = m g h
    \label{eq:ep}
  \end{equation}
  *donde h es distancia hacia abajo ($-y$), no arriba
  \\
  Energía mecánica total
  &
  \begin{equation}
    E_{mt} = E_c + E_c
    \label{eq:emt}
  \end{equation}
  \\
  Principio de la conservación
  &
  \begin{equation}
    E_{mt~i} = E_{mt~f}
    \label{eq:emtemtf}
  \end{equation}
  \medskip

  La fuerza, por lo tanto, es conservativa ($\nabla \times \bar{F} = 0$), por lo cual también tiene una función de potencial.
  \\
  Trabajo
  &
  \begin{equation}
    W = \int \bar{F} \cdot d\bar{r}
    \label{eq:wdef}
  \end{equation}


\end{longtblr}
\end{document}




