%! TeX program = lualatex
%tags! código code hamming
\documentclass[letterpaper]{article}
\usepackage[margin=1in]{geometry}
\usepackage{amsmath}
\usepackage{amssymb}
\usepackage{circuitikz}
\usepackage[no-math]{fontspec}
\usepackage[fg,bg]{gruvboxpalette}
\usepackage{hyperref}
\usepackage{newtxsf}
\usepackage[explicit]{titlesec}
\usepackage{tikz}
\usetikzlibrary{calc}
\usetikzlibrary{positioning}
\usetikzlibrary{arrows.meta}
\usepackage[most]{tcolorbox}
\usepackage{tabularray}
\DefTblrTemplate{firsthead, middlehead,lasthead}{default}{}
\DefTblrTemplate{capcont}{default}{}
\DefTblrTemplate{contfoot-text}{normal}{} \SetTblrTemplate{contfoot-text}{normal} \DefTblrTemplate{conthead-text}{normal}{} \SetTblrTemplate{conthead-text}{normal}
\UseTblrLibrary{counter}
\hypersetup{
  colorlinks  = true,
  urlcolor    = Blue,
  linkcolor   = Blue,
  citecolor   = Blue
}
\usepackage[most]{tcolorbox}

\setmainfont{NotoSans-Regular}[
Path           = /home/snouflake/.fonts/ ,
Extension      = .ttf ,
BoldFont       = NotoSans-Bold ,
ItalicFont     = NotoSans-Italic ,
BoldItalicFont = NotoSans-BoldItalic,
] 

\newtcolorbox{defbox}[3][]{%
  colback=blue!30!background,
  coltitle=blue!15!black,
  coltext=font,
  title filled=false,
	enhanced,
  detach title,
  tile,
  before upper={\tcbtitle\medskip\\},
  borderline west={2mm}{0pt}{blue},
  % attach boxed title to top center={yshift=-2mm},
  leftrule=2mm,
  toprule=0mm,
  bottomrule=0mm,
  rightrule=0mm,
  arc=0mm,
	title={Definición:~#2},
	#1
}

\setlength\parindent{0pt}

\usepackage{mathastext}

\def \T{Sistemas Digitales}
\def \S{Código de paridad}

\begin{document}
\begin{tikzpicture}[inner sep=0pt,color=font]
  \node[anchor=west,align=left,line width=0pt] 
    (title) at (0,0) {\huge\bfseries\noindent\T\\\Large\bfseries\S}
    ;
\end{tikzpicture}


\begin{longtblr}{
    colspec={@{}Q[3cm,cmd=\textbf,h] X@{}},
    rowsep={7pt}
  }
  Código de paridad
  & \begin{minipage}{\linewidth}
  Descubre errores
  \medskip

  $\text{<número binario>}P$ donde $P = 1 \iff$ el número es non.

  \end{minipage}
  \\
  Código de hamming
  & \begin{minipage}{\linewidth}
    $D$ son los digitos del número original.

    $
    D_7
    D_6
    D_5
    P_4
    D_3
    P_2
    P_1
    $ *P incrementa por potencias de 2
    \medskip

    $
    P_1 = 
    D_3 \oplus
    D_5 \oplus
    D_7 \oplus
    D_9 \oplus
    D_{11} \oplus
    \dots
    $ 
    ($D_3$ sí, $P_4$ no, $D_5$ sí, $D_6$ no, $D_7$ sí \dots)
    \medskip

    $
    P_2 = 
    D_3 \oplus
    D_6 \oplus
    D_7 \oplus
    D_{10} \oplus
    D_{11} \oplus
    \dots
    $
    % ($D_3$ sí, $P_4$ no, y de dos en dos- $D_6$ y $D_7$ sí, $D_8$ y $D_9 no$, $D_{10}$ y $D_{11}$ sí \dots)
    (de dos en dos, empezando por $P_2$, el cual no cuenta: $D_3$ sí, $P_4$ y $D_5$ no $D_6$ y $D_7$ sí, $D_8$ y $D_9 no$, $D_{10}$ y $D_{11}$ sí \dots)
    \medskip

    $
    P_4 = 
    D_5 \oplus
    D_6 \oplus
    D_7 \oplus
    D_{12} \oplus
    \dots
    $
    (de cuatro en cuatro, empezando por $P_4$, el cual no cuenta: $D_5, 6, 7$ sí, $8,9,10,11$ no, $12,13,14,15$ sí \dots)
    \medskip

    $
    C_1 = 
    P_1 \oplus
    D_3 \oplus
    D_5 \oplus
    D_7 \oplus
    D_9 \oplus
    D_{11} \oplus
    \dots
    $ 
    \medskip

    $
    C_2 = 
    P_2 \oplus
    D_3 \oplus
    D_6 \oplus
    D_7 \oplus
    D_{10} \oplus
    D_{11} \oplus
    \dots
    $
    \medskip

    $
    C_4 = 
    P_4 \oplus
    D_5 \oplus
    D_6 \oplus
    D_7 \oplus
    D_{12} \oplus
    \dots
    $
    \medskip

    No hay error $\iff \forall C,~ C=0$
  \end{minipage}
\end{longtblr}


 
\end{document}


